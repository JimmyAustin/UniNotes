\documentclass[11pt, oneside]{article}   	% use "amsart" instead of "article" for AMSLaTeX format
\usepackage{geometry}                		% See geometry.pdf to learn the layout options. There are lots.
\geometry{letterpaper}                   		% ... or a4paper or a5paper or ... 
%\geometry{landscape}                		% Activate for for rotated page geometry
%\usepackage[parfill]{parskip}    		% Activate to begin paragraphs with an empty line rather than an indent
\usepackage{graphicx}				% Use pdf, png, jpg, or eps§ with pdflatex; use eps in DVI mode
								% TeX will automatically convert eps --> pdf in pdflatex		
\usepackage{amssymb}

\title{Problem 0 - Assignment 4}
\author{James Austin}
%\date{}							% Activate to display a given date or no date

\begin{document}
\maketitle
\section{Part A}
If L is an undecidable language, and S is any string, then $L \cup \{S\}$ is also undecidable. This is because any algorithm that computes $L \cup \{S\}$ includes the algorithm that computes the language $L$. As the algorithm for $L$ does not exist (as the language is undecidable), this means the algorithm that computes $L \cup \{S\}$ also doesn't exist, and hence the language is undecidable.
\section{Part B}
For a base case, we will use $k = 0$.
\\
$L^{(0)} = L \cup \{\}$
\\
$L^{(0)} = L$

As L is undecidable, this means that $L^{(0)}$ is also undecidable.

For the induction step, we use $k = k+1$\\
$L^{(k+1)} = L \cup {s_1,s_2,s_3,...,s_k} \cup {s_k+1}$\\
As we have shown in Part A, any language that includes a undecidable language in it's definition will also be undecidable. As $L^{(k+1)}$ includes $L$, $L^{(k+1)}$ is also undecidable for all K.

\section{Part C}

The language $L$ is undecidable because there is no algorithm for generating it. There is however an algorithm for generating the set of all strings. The fact that L is a subset of the set of all strings has nothing to do with the algorithm used for generating it.



\end{document}  